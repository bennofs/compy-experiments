\documentclass[169]{beamer}

\usetheme[numbering=fraction]{metropolis}

\usepackage{polyglossia}
\usepackage{csquotes}
\usepackage{fontspec}
\usepackage{blindtext}
\usepackage{graphicx}
\usepackage{listings}
\usepackage{appendixnumberbeamer}
\usepackage{tikz}
\usepackage{pifont}
\usepackage{booktabs}
\usetikzlibrary{positioning}
\usetikzlibrary{arrows.meta}
\usetikzlibrary{shapes.geometric}
\usetikzlibrary{calc}
\usetikzlibrary{fit}
\usetikzlibrary{graphs}
\usetikzlibrary{matrix}
\usepackage{upquote}

\lstset{upquote=true}
\metroset{block=fill}
\setsansfont[BoldFont={Fira Sans SemiBold}]{Fira Sans Book}
\makeatletter
\newlength\beamerleftmargin
\setlength\beamerleftmargin{\Gm@lmargin}
\makeatother

\title{Learning Vulnerability Discovery with Global-Relational Models}
\subtitle{Research Project Compiler Construction}
\author{Benno Fünfstück}
\date{October 7, 2021}

\begin{document}

%\includeonlyframes{current}

\tikzset{
  onslide/.code args={<#1>#2}{%
    \only<#1>{\pgfkeysalso{#2}}% \pgfkeysalso doesn't change the path
  },
  temporal/.code args={<#1>#2#3#4}{%
    \temporal<#1>{\pgfkeysalso{#2}}{\pgfkeysalso{#3}}{\pgfkeysalso{#4}}%
  },
  hidden/.style = {opacity=0},
  uncover/.style = {temporal=#1{hidden}{}{hidden}},
  drawalert/.style = {temporal=#1{}{color=alerted text.fg}{}}
}

\newcounter{tmlistings}

\newcommand\makenode[2]{%
  \tikz[baseline=0pt, remember picture] { \node[fill=gray!50,thick,rounded corners,anchor=base,#1/.try] (listings-\the\value{tmlistings}) {{\scriptsize\the\value{tmlistings}} #2}; }%
  \stepcounter{tmlistings}%
}

\maketitle

\begin{frame}\frametitle{Deep learning for vulnerability detection}
  {\mbox {\lower3.5pt\hbox{\hskip2pt\pgfuseimage{beamericonarticle}\hskip1pt}} Deep Program Structure Modeling Through Multi-Relational Graph-based Learning~\cite{ye_deep_2020}}

  \vspace{10pt}

  \begin{tabular}{lrrr}
    \toprule
    Metrics & µVuldeepecker & Lin et al. & {\bfseries POEM} \\
    \midrule
    C Accuracy & 80.0\% & 88.0\% & \bfseries{90.9\%} \\
    C FPR      & 31.6\% & 30.5\% & \bfseries{3.1\%} \\
    C FNR      & 9.4\%  & \bfseries{7.1\%}  & 8.9\% \\
    \bottomrule
  \end{tabular}

  (Dataset: ``collected from standard vulnerable code databases'')
\end{frame}

\begin{frame}<1>[label=overall]\frametitle{Overall structure: binary classification of functions}
  \begin{tikzpicture}
    \matrix[row sep=20pt, nodes={minimum width=80pt, inner ysep=5pt}] (layers) {
      \node       (inp) {input sample}; \\
      \node[draw,drawalert=<2>] (emb) {embedding}; \\
      \node[draw,drawalert=<3>] (rep) {representation}; \\
      \node[draw,drawalert=<4>] (ext) {extraction}; \\
      \node       (out) {prediction}; \\
    };

    \node[above right=10pt and 70pt of rep, text width=130pt] (seq-rnn) {sequence-based RNN};
    \node[right=10pt and 70pt of rep, text width=130pt] (graph-ggnn) {graph-based GGNN \cite{li_gated_2017}};
    \node[below right=10pt and 70pt of rep, text width=130pt] (combined-sandwich) {sandwich model \cite{hellendoorn_global_2019}};

    \node[right=70pt of inp] (functions) {\bfseries functions};
    \node[right=70pt of out,text width=95pt] (vuln) {\bfseries \hfill not vulnerable (0)\\ \hfill vulnerable (1)};

    \graph { (inp) -> (emb) -> (rep) -> (ext) -> (out) };

    \draw[dashed]
      (rep.east) -- (seq-rnn.north west) -- ({$(combined-sandwich.south east)$} |- {$(seq-rnn.north east)$}) -- (combined-sandwich.south east) -- (combined-sandwich.south west) -- cycle;
    \draw[dashed, <-] (inp) -- (functions);
    \draw[dashed, ->] (out) -- (vuln);
  \end{tikzpicture}
\end{frame}


\begin{frame}\frametitle{Are sandwich models better for vulnerability prediction?}
  \begin{exampleblock}{Goal}
    Evaluate the sandwich model from Hellendoorn et al. on the task of vulnerability detection
  \end{exampleblock}

  \begin{enumerate}
    \item Implement in Compy-Learn~\cite{brauckmann_compy-learn_2020}
    \item Review existing datasets for ML vulnerability discovery
    \item Evaluate the model on \emph{real world data}
  \end{enumerate}
\end{frame}

\section{Model}

\begin{frame}[fragile]\frametitle{A sample function}

\begin{lstlisting}[language=C, morekeywords={uint8_t},
  emph={buf}, emphstyle=\color{green!60!black},
  emph={[2]out}, emphstyle={[2]\color{blue!90!black}},
  emph={[3]pkt_len}, emphstyle={[3]\color{cyan!80!black}},
  emph={[4]i}, emphstyle={[4]\color{alert}},
]
int decode_packet(char *buf, char *out) {
  uint8_t pkt_len = buf[0];
  for (int i = 0; i < pkt_len; ++i) {
    out[i] = buf[i + 1];
  }
  return pkt_len;
}
\end{lstlisting}

  \vspace{10pt}

  \begin{tikzpicture}[
    keyword/.style = {font=\bfseries},
    buf/.style = { text=green!60!black },
    uncover=<2->
    ]
    \matrix[matrix of nodes, column sep=5pt, row sep=20pt, ampersand replacement=\&, nodes={
      draw, minimum height=20pt, anchor=north, text height=10pt, text depth=3pt,
    }] {
      |[keyword] (n0)|int \& |(n1)| decode\_packet \& |[inner xsep=23pt] (n2)| ( \& |[keyword] (n3)|char \& |[inner xsep=10pt] (n4)| * \& |[buf] (n5)|buf \& |[draw=none]| {\large$\cdots$}  \\
    };
    \node[anchor=north west, inner xsep=0pt] at (n0.south west) { int };
    \node[anchor=north west, inner xsep=0pt] at (n1.south west) { identifier };
    \node[anchor=north west, inner xsep=0pt] at (n2.south west) { l\_paren };
    \node[anchor=north west, inner xsep=0pt] at (n3.south west) { char };
    \node[anchor=north west, inner xsep=0pt] at (n4.south west) { star };
    \node[anchor=north west, inner xsep=0pt] at (n5.south west) { identifier };

  \end{tikzpicture}
\end{frame}

\begin{frame}[t]\frametitle{Recurrent Neural Networks (RNN)}
  \begin{tikzpicture}[
    keyword/.style = {font=\bfseries},
    buf/.style = { text=green!60!black },
    ]
    \matrix[matrix of nodes, column sep=5pt, row sep=0pt, ampersand replacement=\&, nodes={
      draw, minimum height=20pt, anchor=north, text height=10pt, text depth=3pt,
    }] {
      |[keyword] (n0)|int \& |(n1)| decode\_packet \& |[inner xsep=23pt] (n2)| ( \& |[keyword] (n3)|char \& |[inner xsep=10pt] (n4)| * \& |[buf] (n5)|buf \& |[draw=none]| {\large$\cdots$}  \\
      |[draw=none]| int   \& |[draw=none]| identifier \& |[draw=none]| l\_paren \& |[draw=none]| char \& |[draw=none]| star \& |[draw=none]| identifier \& \\
    };

    % \node[anchor=north west, inner xsep=0pt] at (n0.south west) { int };
    % \node[anchor=north west, inner xsep=0pt] at (n1.south west) { identifier };
    % \node[anchor=north west, inner xsep=0pt] at (n2.south west) { l\_paren };
    % \node[anchor=north west, inner xsep=0pt] at (n3.south west) { char };
    % \node[anchor=north west, inner xsep=0pt] at (n4.south west) { star };
    % \node[anchor=north west, inner xsep=0pt] at (n5.south west) { identifier };


  \end{tikzpicture}
\end{frame}

\begin{frame}\frametitle{Recurrent Neural Networks (RNN)}
\end{frame}

\begin{frame}\frametitle{Representation as Graph: augmented AST}
\end{frame}

\begin{frame}\frametitle{Gated Graph Neural Networks (GGNN)}
\end{frame}

\begin{frame}\frametitle{Sandwich models: Combining GGNNs and RNNs}
\end{frame}

\begin{frame}\frametitle{Batching for sandwich models}
\end{frame}

\begin{frame}\frametitle{Extraction: Global Attentation}
\end{frame}

\section{Data}

\begin{frame}\frametitle{Real-World Vulnerabilities: CVE database}
  \textbf{Common Vulnerabilities and Exposures (CVE)} \\
  Community catalog of publicly disclosed cybersecurity vulnerabilities

  \textbf{Common Weakness Enumeration (CWE)} \\
  Community-developed list of software and hardware weakness types

  \textbf{National Vulnerability Database (NVD)} \\
  U.S. government repository of standards based vulnerability management data (includes CVE data)
\end{frame}

\begin{frame}[label=current]
  \only<1>{
  \includegraphics[width=\framewidth]{media/cve-2014-detail}
  }
  \only<2>{
  \includegraphics[width=\framewidth]{media/cve-2014-refcwe}
  }
\end{frame}

% ReVeal
% Devign
% Draper
% SARD/SATE
%

\section{Results}

% no model can fit the dataset
% overfitting is possible, but no generalization
% can detect whether it is likely to be involved in vulnerabilities, but not if it has been fixed or not


\appendix

\begin{frame}[allowframebreaks]{References}
  \bibliography{zotero}
  \bibliographystyle{apalike}
\end{frame}


\end{document}
