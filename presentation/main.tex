\documentclass[169]{beamer}

\usetheme[numbering=fraction]{metropolis}

\usepackage{polyglossia}
\usepackage{csquotes}
\usepackage{fontspec}
\usepackage{blindtext}
\usepackage{graphicx}
\usepackage{listings}
\usepackage{appendixnumberbeamer}
\usepackage{tikz}
\usepackage{pifont}
\usetikzlibrary{positioning}
\usetikzlibrary{arrows.meta}
\usetikzlibrary{shapes.geometric}
\usetikzlibrary{calc}
\usetikzlibrary{fit}
\usepackage{upquote}

\lstset{upquote=true}
\metroset{block=fill}
%\setsansfont[BoldFont={Fira Sans SemiBold}]{Fira Sans Book}
\makeatletter
\newlength\beamerleftmargin
\setlength\beamerleftmargin{\Gm@lmargin}
\makeatother

\title{Learning Vulnerability Discovery with Global-Relational Models}
\subtitle{Research Project Compiler Construction}
\author{Benno Fünfstück}
\date{October 7, 2021}

\begin{document}

\includeonlyframes{current}

\tikzset{
  onslide/.code args={<#1>#2}{%
    \only<#1>{\pgfkeysalso{#2}}% \pgfkeysalso doesn't change the path
  },
  temporal/.code args={<#1>#2#3#4}{%
    \temporal<#1>{\pgfkeysalso{#2}}{\pgfkeysalso{#3}}{\pgfkeysalso{#4}}%
  },
  hidden/.style = {opacity=0},
  uncover/.style = {temporal=#1{hidden}{}{hidden}},
}

\maketitle

% 3 min
\begin{frame}\frametitle{Intro}
  \begin{columns}
    \begin{column}{0.5\textwidth}
      \includegraphics[height=0.85\textheight]{./media/bug-captcha}
    \end{column}
    \begin{column}{0.5\textwidth}
      \begin{enumerate}
        \item Approach
        \item Model
        \item Data
        \item Results
        \item Summary
      \end{enumerate}
    \end{column}
  \end{columns}
\end{frame}

\begin{frame}[fragile]\frametitle{Goal: Predict vulnerable functions}
  % begin center somehow breaks the alignment of the lstlisting blocks
  \hbox to \hsize{\hfil{%
  \begin{tikzpicture}
    \node[uncover=<1>, anchor=north west] (code1) {
\begin{lstlisting}[language=C]
int readNumber(void) {
  char buf[32];
  gets(buf);
  buf[31] = '\0';
  return atol(buf);
}
\end{lstlisting}
      };

      \node[uncover=<2>, anchor=north west] (code2) at (code1.north west) {
\begin{lstlisting}[language=C]
int readNumber(void) {
  char buf[32];
  fgets(buf, 32, stdin);
  buf[31] = '\0';
  return atol(buf);
}
\end{lstlisting}
      };

      \node[uncover=<3>, anchor=north west] (code3) at (code1.north west) {
\begin{lstlisting}[language=C]
int readNumber(void) {
  char buf[32];
  fgets(buf, 42, stdin);
  buf[31] = '\0';
  return atol(buf);
}
\end{lstlisting}
      };

      \node[uncover=<4>, anchor=north west] (code4) at (code1.north west) {
\begin{lstlisting}[language=C]
int readNumber(void) {
  char buf[32];
  fgets(buf, sizeof(buf), stdin);
  buf[31] = '\0';
  return atol(buf);
}
\end{lstlisting}
      };

      \node[red, below=40pt,onslide=<2>hidden,onslide=<4>hidden] (vuln) at (code1.south) { vulnerable };
      \node[green!60!black, below=40pt, text width=75pt,onslide=<1>hidden,onslide=<3>hidden] (patched) at (code1.south) { not vulnerable \\ (patched) };

      \path[draw,onslide=<2>hidden,onslide=<4>hidden] (code1) edge[->] node[left=7pt]{\emph{model}} (vuln);
      \path[draw,onslide=<1>hidden,onslide=<3>hidden] (code1) edge[->] node[left=7pt]{\emph{model}} (patched);

  \end{tikzpicture}%
}\hfil}
\end{frame}

\section{Model}
% preprocessing -> embedding -> propagation -> prediction

\section{Data}

\begin{frame}\frametitle{Real-World Vulnerabilities: CVE database}
  \textbf{Common Vulnerabilities and Exposures (CVE)} \\
  Community catalog of publicly disclosed cybersecurity vulnerabilities

  \textbf{Common Weakness Enumeration (CWE)} \\
  Community-developed list of software and hardware weakness types

  \textbf{National Vulnerability Database (NVD)} \\
  U.S. government repository of standards based vulnerability management data (includes CVE data)
\end{frame}

\begin{frame}[label=current]
  \only<1>{
  \includegraphics[width=\framewidth]{media/cve-2014-detail}
  }
  \only<2>{
  \includegraphics[width=\framewidth]{media/cve-2014-refcwe}
  }
\end{frame}

% ReVeal
% Devign
% Draper
% SARD/SATE
%

\section{Results}

% no model can fit the dataset
% overfitting is possible, but no generalization
% can detect whether it is likely to be involved in vulnerabilities, but not if it has been fixed or not

\end{document}
